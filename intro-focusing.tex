%!TEX root = main.tex
% !TEX spellcheck = en-US

Suppose that, in intuitionistic logic, we would like to try to prove the following sequent in intuitionistic logic
\[
A\iimp B, C\wedge D\seq p
\]
where $p$ is an atomic formula. There are two ways of proceeding with proof search (hence in a bottom-up reading of rules): either apply the conjunction left rule or the implication left rule.
%many ways of doing so. For example, we could start by applying a rule over the main connective of $F$, or one of the formulas in $\Gamma$, if they exist. Another possibility would be start with 

If we decide for the first, assuming (the multiplicative version of) the conjunction left rule, the derivation would look like
\[
\infer[\wedge_{l_M}]{A\iimp B, C\wedge D\seq p}
{\deduce{A\iimp B, C, D\seq p}{
\deduce{}{\vdots}}}
\] 
Since $(\wedge_{l_M})$ is an invertible rule~\cite{troelstra00book}, its application does not affect provability in the sense that, if the conclusion is provable, so is the premise.

Now, if instead we decide to apply the left rule for implication\footnote{We will avoid the discussion on copying the  implication to the left premise here. See~\cite{DBLP:journals/jsyml/Dyckhoff92}.}
\[
\infer[\iimp_{l}]{A\iimp B, C\wedge D\seq p}{C\wedge D\seq A & B, C\wedge D\seq p}
\] 
then provability may be lost, since $(\iimp_{l})$ is not invertible w.r.t. the left premise, that is, it may be the case that the conclusion sequent is provable but the left premise sequent is not.

Hence, when  searching for a proof of $A\iimp B, C\wedge D\seq p$ we can {\em always} start by applying the $(\wedge_{l_M})$ --  {\em don't care non-determinism} -- postponing making decisions, like applying $(\iimp_{l})$, until there are no more invertible options left. At that point, a decision has to be made: which non-invertible step should be taken --  {\em don't know non-determinism}. 

This is the essence of focusing~\cite{andreoli92jlc}: separate the proof steps into unfocused (no changes in provability) and focused (decisions must be made). In the focused step, once we decide to {\em focus on} (or work on) a formula, this focus is maintained, bottom-up, after the application of the rule. We mark the focused formulas with the downarrow symbol $\Downarrow$. The focused left rule for the implication is
\[
\infer[\iimp_{l}]{\jLf{\Gamma}{A\iimp B}{C}}{\jRf{\Gamma}{A}{} & \jLf{\Gamma}{B}{C}}
\] 
We read this as: if we decide to decompose $A\iimp B$ on the left, we should continue the proof search by maintaining the focus  on $A$ on the right and $B$ on the left.

But what about {\em atomic formulas}? Suppose that $A,B$ are atomic formulas in the focused derivation above. It is
        possible to impose two different ``protocols'' for dealing with the atomic case.  The $Q$-protocol insists that the left premise above is trivial, meaning that it is proved by
        the initial rule.  On the other hand, focusing should be lost (represented by the uparrow $\Uparrow$) in the right premise
\[
\infer[\iimp_{l}]{\jLf{\Gamma}{A\iimp B}{C}}
{\infer[\kinit]{\jRf{\Gamma}{A}{}}{} & 
\infer{\jLf{\Gamma}{B}{C}}{\jUnf{\Gamma}{B}{C}{}}}
\]  
        Following that protocol, we have that it should be the case that
        $A\in\Gamma$.  Thus, if we set $\Gamma'$
        to be the result of removing all occurrences of $A$
        from $\Gamma$, then the (unfocused) derived inference rule from the derivation above becomes
\[
  \infer[\iimp_{l_Q}]
        {\Gamma', A, A\iimp B \seq C}
        {\Gamma' , B\seq C}
\]
        The second protocol, the $T$-protocol, insists that the right-most
        premise is trivial and focus should be lost in the left premise
\[
\infer[\iimp_{l}]{\jLf{\Gamma}{A\iimp B}{C}}
{\infer{\jRf{\Gamma}{A}{}}{\jUnf{\Gamma}{}{A}{}} & 
\infer[\kinit]{\jLf{\Gamma}{B}{C}}{}}
\]   
That is,
        $B$ and $C$ are the same atomic formula.       
Thus,  the (unfocused) derived inference rule from the derivation above becomes
\[
  \infer[\iimp_{l_T}]
        {\Gamma, A\iimp B \seq B}
        {\Gamma \seq A}
\]
The names for the $Q$ and $T$ protocols comes from Danos, Joinet, and
Schellinx~\cite{danos93wll}: in the $Q$ protocol, the tail (``queue'')
of an implication yields a trivial premise while in the $T$ protocol,
the head (``t\^ete'') of an implication yields a trivial premise.

A more modern and flexible presentation of the $Q$ and $T$ protocols
speaks, instead, of the \emph{polarity} of formulas.
%
In particular, if all atomic formulas have a {\em positive polarity}, the
$Q$-protocol is enforced, while if all atomic formulas have a {\em negative
polarity}, the $T$-protocol is enforced.

Now that focusing is understood for implication and atoms, what can be said about the other connectives? For example, in the case of conjunction, we adopted the (invertible) {\em multiplicative version} of the left rule, that is
\[
\infer[\wedge_{l_M}]{\Gamma, A\wedge B\seq C}{\Gamma, A, B\seq C}
\] 
%This rule incorporates the left contraction rule. Gentzen's original rules correspond to 
The (non-invertible) {\em additive version} is the following, where a choice has to be made during proof search
\[
 \infer[\wedge_{l_{i}}]{A_1\wedge A_2, \Gamma \seq C}{A_i, \Gamma \seq C}
\]
Such multiplicative/additive, invertible/non-invertible flavors can be captured in a single proof system by splitting the conjunction into two connectives: 
$\wedgep$ and $\wedgen$, with {\em unfocused/focused} left rules 
%Of course we have $A\wedgen B$ is logically equivalent to $A\wedgep B$, so while provability is maintained, the form of the resulting proofs can be completely different.
\[
  \infer[\wedgep_{l}]{\jUnf{\Gamma}{A\wedgep B,\Theta}{}{C}}{\jUnf{\Gamma}{A , B, \Theta}{}{C}}
  \qquad
 \infer[\wedgen_{l_i}]{\jLf{\Gamma}{A_1 \wedgen A_2}{C}}{\jLf{\Gamma}{A_i}{C}}
 \]


In \cite{liang07csl,LiaMil09} Miller and Liang proposed the  \LKF and \LJF focused proof
systems for classical and intuitionistic logics, respectively.
%
Those systems 
extend both the notion of focusing and polarity to
all formulas. 

In such systems, {\em focused rule applications} imply that focus is transferred from conclusion to premises in derivations. This process goes on until either the focused phase ends (depending on the {\em polarity} of the focused formula), or the derivation ends.
Once the focus is \emph{released}, the formula is eagerly decomposed into subformulas, which are ultimately {\em stored} in the context. 
%
We will describe this in detail in the next section.

{\em Some historical remarks.} The focusing discipline is based on the notion of {\em uniform proofs}~\cite{miller91apal} and it was discovered by Andreoli in~\cite{andreoli92jlc}, who showed that it is complete for linear logic~\cite{DBLP:journals/tcs/Girard87}, which naturally contains the multiplicative and additive versions of disjunction and conjunction. In intuitionistic logic focusing gives rise to  call-by-value~\cite{dyckhoff06cie}  and call-by-name~\cite{herbelin94csl} calculi, since using the $Q$-protocol the proof-search semantics of the
implication is given by 
\emph{forward-chaining}, while using the $T$-protocol, the proof-search semantics of the
 implication is given by
\emph{back-chaining}.
