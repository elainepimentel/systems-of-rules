\documentclass[runningheads]{llncs}

\usepackage{xspace}
\usepackage{proof}
\usepackage{xcolor}
\usepackage{amsmath}
\usepackage{amssymb}
\usepackage{cmll}
\usepackage[colorlinks,linkcolor=blue,citecolor=blue]{hyperref}
\usepackage{xifthen}

\newcommand{\lP}{$\lambda$Prolog\xspace}
\newcommand{\ie}{i.e.,\xspace}
\newcommand{\eg}{e.g.,\xspace}
\newcommand{\red}[1]{\textcolor{red}{#1}}
\newcommand{\green}[1]{\textcolor{green}{#1}}
\newcommand{\blue}[1]{{\color[rgb]{0,0,1} #1}}

\newcommand{\DM}[1]{\smallskip\par\noindent\red{DM #1}\smallskip}
\newcommand{\EP}[1]{\smallskip\par\noindent\red{EP #1}\smallskip}

\newcommand\proofsystem[1]{\mbox{\slshape #1}\xspace}
\newcommand\LK   {\proofsystem{LK}}
\newcommand\LJ   {\proofsystem{LJ}}
%% \newcommand\LKT  {\proofsystem{LKT}}
%% \newcommand\LKQ  {\proofsystem{LKQ}}
%% \newcommand\LKF  {\proofsystem{LKF}}
%% \newcommand\aLKF {\hbox{\proofsystem{LKF}\kern-2pt$^a$}\xspace}
%% \newcommand\LKU  {\proofsystem{LKU}}
%% \newcommand\LJT  {\proofsystem{LJT}}
%% \newcommand\LJQ  {\proofsystem{LJQ}}
%% \newcommand\LJF  {\proofsystem{LJF}}
%% \newcommand\aLJF {\hbox{\proofsystem{LJF}\kern-2pt$^a$}\xspace}
%% \newcommand\LKpos{\proofsystem{LKpos}}
%% \newcommand\LKneg{\proofsystem{LKneg}}
%% \newcommand\negclass[2]{\mathcal{N}_{#1}^{#2}}
%% \newcommand\posclass[2]{\mathcal{P}_{#1}^{#2}}

%% \newcommand{\tup}[1]{\langle #1\rangle}

\newcommand{\false}{f\/}
\newcommand{\imp}{\supset}
\newcommand{\nfalse}{f^-}
\newcommand{\ntrue}{t^-}
\newcommand{\pfalse}{f^+}
\newcommand{\ptrue}{t^+}
\newcommand{\ra}{\rightarrow}
\newcommand{\true }{t\/}
\newcommand{\veen}{\vee^{\!-}}
\newcommand{\veepn}{\vee^{\!\pm}}
\newcommand{\veep}{\vee^{\!+}}
\newcommand{\wedgen}{\wedge^{\!-}}
\newcommand{\wedgepn}{\wedge^{\!\pm}}
\newcommand{\wedgep}{\wedge^{\!+}}

\newcommand{\twoseq}[2]{#1\vdash #2}
\newcommand{\Twoseq}[3]{#1:#2\vdash #3}
\newcommand{\R}[2]{R #1 #2}

\begin{document}
\title{
%      A system of rules for sequent calculus\\(an abstract)\\
%      --- or ---\\
       Higher-level rules for sequent calculus\\(an abstract)
}
\author{Dale Miller\inst{1}%\orcidID{0000-0003-0274-4954}
        \and
        Elaine Pimentel\inst{2}%\orcidID{0000-0002-7113-0801}
}
\institute{
  Inria-Saclay \& LIX, Ecole Polytechnique, Palaiseau, France\\
%  \email{dale.miller@inria.fr}
  \and
  Dept. of Computer Science, University College London, UK \\
%  \email{e.pimentel@ucl.ac.uk}
}

\maketitle              

Theories over a base logic can be represented in two different ways:
as collections of formulas accepted as axioms and assumptions or as
collections of inference rules that do not involve logical
connectives.

Schroeder-Heister~\cite{schroeder-heister84} and Avron
\cite{avron90ndjfl} showed the equivalence in propositional
intuitionistic logic between conventional natural deduction (where
formulas can be assumed and discharged) and natural deduction with
higher-level rules (where atomic formulas and rules can be assumed and
discharged).  The LF \cite{harper93jacm} specification framework
generalized this work on natural deduction to quantificational
intuitionistic logic while simultaneously encoding the extended
natural deduction proofs using a term representation within a
dependently typed $\lambda$-calculus.

Moving from natural deduction to sequent calculus permits developing
approaches to higher-level rules for both classical and intuitionistic
logic proofs~\cite{ciabattoni18tocl,negri03aml}. In particular, it is
natural to ask whether or not an \LJ or \LK proof of the sequent
$\twoseq{\Gamma}{\Delta}$ can be built where the formulas in $\Gamma$
(the assumptions) are replaced with inference rules. It was shown in
\cite{negri03aml} that this replacement is possible when assumptions
are \emph{geometric formulas}. For example, if the multiset of
assumptions is $\Gamma, \forall x\forall y\forall
z.[\R{x}{z}\imp\R{z}{y}\imp\R{x}{y}]$ then the assumption stating the
transitivity of the predicate $R$ can be replaced by one of the
following inference rules.
\[
  \infer[\hbox{backchain}]
        {\twoseq{\Gamma}{\R{x}{y}}}
        {\twoseq{\Gamma}{\R{x}{z}} & \twoseq{\Gamma}{\R{z}{y}}}
  \qquad
  \infer[\hbox{forwardchain}]
        {\twoseq{\Gamma,\R{x}{z},\R{z}{y}}{B}}
        {\twoseq{\Gamma,\R{x}{z},\R{z}{y},\R{x}{y}}{B}}
\]
In this way, we have replaced a formula containing five occurrences of
logical connectives with one of two rules, neither containing logical
connectives. If the only assumptions are, say, Horn clauses, it is
possible to have a complete proof system for atomic consequences of
such clauses using a proof system involving only atomic formulas.

While it is possible to generalize the restriction to geometric
formulas a bit to \emph{bipoles} by considering
polarity~\cite{marin22apal}, it is interesting to ask to what extent
can non-bipole formulas be replaced with higher-level rules in the
sequent setting.  Such a question was addressed in \cite{negri16jlc},
where \emph{generalized geometric formulas} are treated using a
\emph{system of rules}, a setting in which an inference rule can allow
some of its premises to have additional inference rules available.
These additional inference rules are scoped over particular proofs of
premises.  We will show that such a scoping of inference rules is a
direct reading of inference rules in a polarized proof system.  For
example, focusing on the (polarized) formula that states the
existence of least upper bounds
\[
  \forall x\forall y\exists z(x\le z\wedgep y\le z\wedgep
              \forall w(x\le w\wedgep y\le w\imp z\le w)),
\]
yields the synthetic inference rule (terminology from
\cite{marin22apal})
\[
  \infer[(*).]
        {\Twoseq{\Sigma}{\Gamma}{\Delta}}
        {\Twoseq{\Sigma,z}
                {\forall w(x\le w\wedgep y\le w\imp z\le w),
                   x\le z,y\le z,\Gamma}
                {\Delta}}
\]
Here, sequents are prefixed with a list of variables, \eg $\Sigma$,
that are the eigenvariables that may appear in the formulas of that
sequent: this prefix is intended to \emph{bind} variables over the
sequent it precedes.  Note that the prefix of the conclusion is
different from the prefix of the premise.  The availability of the
additional assumption $\forall w(x\le w\wedgep y\le w\imp z\le w)$ in
the premise corresponds exactly to having either one of the following
two inference rules scoped over that premise
\[
  \infer
        {\Twoseq{\Sigma,z}{\Gamma}{z\le w}}
        {\Twoseq{\Sigma,z}{\Gamma}{x\le w} & \twoseq{\Gamma}{y\le w}}
  \qquad
  \infer[,]
        {\Twoseq{\Sigma,z}{\Gamma,x\le w,y\le w}{\Delta}}
        {\Twoseq{\Sigma,z}{\Gamma,x\le w,y\le w, z\le w}{\Delta}}
\]
depending on whether or not the polarity of the $\le$ predicate is
negative or positive, respectively.  Thus, we can rewrite the
inference figure $(*)$ so that it does not mention any logical
connectives by stipulating that one of these two inference rules is
available to prove that premise.

We will also point out how Tseitin predicate symbols, often introduced
into formulas to reduce their logical complexity~\cite{blair86cor}, can
be used to make scoped rules available globally.

\bibliography{../../../Dropbox/Writing/references/master}
% Switch the following with the line above if master.bib is not available
%\bibliography{extract,local} % Do not edit extract.bib.  Add to local.bib.

\bibliographystyle{splncs04}
\end{document}

% Formulas have their advantages: easier to formalize in proof
% assistants and they are more compact wrt non-determinism (dnf/cnf is
% needed generally to make inference rules).

% LocalWords:  Avron forwardchain Tseitin
