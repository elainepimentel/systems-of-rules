\documentclass[9pt]{beamer}

\input macros.tex

\title{Higher-level rules for sequent calculus}
\author[Miller, Pimentel]{
  \emph{Dale Miller} \\
  {\small \'Ecole Polytechnique, France} \\[20pt]
  {\small Joint work with Elaine Pimentel} \\[20pt]
  {\small 6th International Workshop on Proof Theory, Birmingham}
}
\date{12 September}

\begin{document}
\begin{frame}
	\titlepage
\end{frame}	


\begin{frame}{The original axioms-as-rules problem}

\begin{overlayarea}{\textwidth}{7cm}	
\only<1->{	
	How to incorporate \emphdr{inference rules} encoding axioms into existing proof systems 
	
	for \emphdb{classical and intuitionistic logics}?
}

\only<2-3>{
\medskip

\emphdb{Gentzen:} Add mathematical theories to first-order logic.

\begin{center}
{\em Consistency of the arithmetic without complete induction.}
\end{center}

\begin{tabular}{lcl}
\includegraphics[scale=0.5]{figs/gentzen1} &\qquad& \includegraphics[scale=0.3]{figs/gentzen3}\\
\includegraphics[scale=0.5]{figs/gentzen2} 
\end{tabular}
}	
\only<3>{
\begin{center}
{\em ``If our arithmetic is inconsistent, there exists a [cut-free] $\LK$ derivation with endsequent
\[
\mathfrak{U}_1,\ldots \mathfrak{U}_n\seq
\] 
where $\mathfrak{U}_1,\ldots \mathfrak{U}_n$ are arithmetic axiom formulae.''}
\end{center}
}


\only<4-5>{
\medskip

\emphdo{A naive attempt:} Add non-logical axioms.

\medskip

Assume $\vdash P\impl Q$ and $\vdash P$. 
			Then\semiproofadjust
			$$
			\infer[cut]{\vdash Q}
			{\infer{\vdash P}{}&
				\infer[cut]{P\vdash Q}
				{\infer{\vdash P\impl Q}{}&
					\infer[\impl_l]{P,P\impl Q\vdash Q}
					{\infer{P\vdash P}{}&
						\infer{Q\vdash Q}{}}}}
			$$
}
\only<5>{
\medskip

\emphdb{Girard}: The {\em Hauptsatz} fails for systems with proper axioms.}

\only<6-7>{
\medskip

\emphdo{A better solution:} Add non-logical rules of inference


\medskip
			
			$$
			\infer[\emphdb{P\impl Q}]{\Gamma, P\seq C}{\Gamma, Q\seq C}\qquad
			\infer[\emphdb{P}]{\Gamma\seq C}{\Gamma, P\seq C}
			$$
		}	
		
		\only<7>{
						The sequent $\seq Q$ now has the (cut-free) proof\semiproofadjust
			$$
			\infer[\emphdb{P}]{\seq Q}{\infer[\emphdb{P\impl Q}]{P\seq Q}
				{\infer{Q\seq Q}{}}}
			$$
		}
\end{overlayarea}
\end{frame}

\section{Polarisation}
\begin{frame}
\frametitle{Polarities of connectives}

\textbf{{Polarization is a feature of linear logic: $\otimes$, $\&$,
    $\oplus$, $\parr$}}
\begin{itemize}
  \item If the right-introduction rule is invertible, the connective
    is \negf{negative}.
  \item If the left-introduction rule is invertible, the connective
    is \posf{postive}.
  \item De Morgan duality flips polarity.  Polarity for atoms is
    assigned arbitrarily.
\end{itemize}
\vfill
\pause

\textbf{First-order classical and intuitionistic language:}
$$A\coloncolonequals P(x) \mid A \wedge A \mid t \mid A \vee A 
        \mid f \mid A \impl A \mid \exists x\, A \mid \forall x\, A$$
\vfill

\textbf{{Polarized connectives:}}
\begin{itemize}
\item In \emphdr{classical  logic}%, the \emph{polarized connectives} are:
\begin{itemize}
\item \posf{positive} and \negf{negative} versions of the logical connectives and constants: 
\[\wedgen, \wedgep, \truen, \truep,\veen, \veep, \falsen, \falsep\]
\end{itemize}
\item In \emphdr{intuitionistic logic} 
\begin{itemize}
\item polarized classical connectives and constants where $\falsen,\veen$ do not occur;
\item \negf{negative} implication: $\impl$.
\end{itemize}
\item First-order quantifiers: $\forall$ \negf{negative} and $\exists$ \posf{positive}.
\item A formula is \posf{positive} if it is a positive atom or has a
top-level positive connective.
\item A formula is \negf{negative} if it is a
negative atom or has a top-level negative connective.
\end{itemize}
\end{frame}

\begin{frame}{A fresh view to an old problem}

\begin{overlayarea}{\textwidth}{7cm}			
 \only<1->{Combining the \emphdb{polarities' hierarchy} [Ciabattoni et al.] with a}
 \only<1>{\medskip
          \begin{center}
          \includegraphics[scale=0.8]{figs/h5}
          \end{center}}
 \only<1>{\[\hbox{e.g., if $B$ is in $\posf{{\mathcal P}_0}$ then }
            \forall x\exists y\forall z. B\hbox{ is in }\negf{{\mathcal N}_3}.\]
  \medskip
  {\small $(\forall x\posf{P_1}\wedgen\posf{P_2})\wedgen(\forall y{B(y)}\wedgen \posf{P_3})$}
  \qquad
  \parbox[c]{2.5cm}{
     \scalebox{.65}{
       \begin{tikzpicture}[sibling distance=7em,level distance=4ex, level 2/.style={sibling distance =4em}]
           \node (neg 1) {$\wedgen$\strut}
		child { node {\rlap{$\wedgen$}\phantom{$\vee$}}
			child { node {$\forall$}
				child[level distance=3ex] {node[itria] {\smash{\posf{$P_1$}}}}}
			child {
					child[level distance=1ex] {node[itria] {\smash{\posf{$P_2$}}}}}}
			child { node {\rlap{$\wedgen$}\phantom{$\vee$}}
				child { node {$\forall$}
					child[level distance=5.5ex] {node {${B(y)}$}}}
				child {
					child[level distance=1ex] {node[itria] {\smash{\posf{$P_3$}}}}}};
      \end{tikzpicture}}}
   $\qquad\bm\rightarrow\quad$
	\parbox[c]{2.5cm}{
	\scalebox{.65}{
		\begin{tikzpicture}[sibling distance=4em,level distance=3.2ex]
		\node[ellipse,draw = orange,fill = orange!10] (neg2) {\negf{$\mathsf{neg}$}}
		child {
			child [level distance=-1ex]{node[itria] {\smash{\posf{$P_1$}}}}}
		child {
			child [level distance=-1ex]{node[itria] {\smash{\posf{$P_2$}}}}}
		child [level distance=6ex]{node {{$B(y)$}}}
		child {
			child [level distance=-1ex]{node[itria] {\smash{\posf{$P_3$}}}}};
		\end{tikzpicture}}}
  \medskip
  \begin{center}
  \emphdr{Bipolar} = $\mathcal{N}_2$\\
  (polarities flip at most twice)
  \end{center}}

\only<2->{
\medskip
systematic construction of synthetic rules from axioms using
\emphdb{focusing} [Andreoli 1912],}
\only<2>{
\medskip
	\begin{tikzpicture}
	\node at (-1.9,2.5) {$\jUnf{\Gamma,B,\Gamma_1}{\cdot}{\cdot}{\Delta_1}\,\cdots\,\jUnf{\Gamma,B,\Gamma_n}{\cdot}{\cdot}{\Delta_n}$};
	\node at (-1.5,-1.4) {\kern -30pt\raisebox{8pt}{$(B\in\Gamma)~~$}$\infer[\kern -2pt D_l]{
			\jUnf{\Gamma}{\cdot}{\cdot}{\Delta}}{
			\jLf{\Gamma}{B}{\Delta}}$};
	\node at (0,0) {\includegraphics[scale=0.65]{figs/focusing}};
		\end{tikzpicture}}
\only<3->{\medskip\emph{justifies} the introduction of the class of
          \emphdr{bipolar axioms}. Here, $B\in\mathcal{N}_2$.}
\only<3>{\medskip
\begin{tikzpicture}
	\node at (-1.9,2.5) {$\jUnf{\Gamma,B,\Gamma_1}{\cdot}{\cdot}{\Delta_1}\;\ldots\;\jUnf{\Gamma,B,\Gamma_n}{\cdot}{\cdot}{\Delta_n}$};
	\node at (-1.5,-1.4) {$\infer[\kern -2pt D_l]{
			\jUnf{\Gamma,B}{\cdot}{\cdot}{\Delta}}{
			\jLf{\Gamma,B}{B}{\Delta}}$};
        \node at (-2.2,0.7) {\includegraphics[scale=0.65]{figs/focusing2}};
	\node at (4,1) {\textbf{Corresponding synthetic rule}};
	\node at (4,0.5) {(in $\LK$ or $\LJ$)};
	\node at (4,-0.5) {
          $\infer[B]{\Gamma\seq\Delta}
                    {\Gamma,\Gamma_1\seq\Delta_1\quad\ldots\quad\Gamma,\Gamma_n\seq\Delta_n}$};
\end{tikzpicture}}
\end{overlayarea}
If $\Gamma\subseteq\mathcal{N}_{n}$  then
$\Gamma_i\subseteq\mathcal{N}_{n-2}$, for all $i=1,\ldots,n$.
\end{frame}

\begin{frame}
\frametitle{The main results [Marin, Miller, Pimentel \& Volpe, 2022]}	
	
	\begin{block}{}
		\textbf{Theorem 1.} 
		Synthetic rules built from bipolar axioms 
                \emph{involve only atomic formulas}.
	\end{block}
	
	\bigskip
	
	\begin{block}{}
		\textbf{Theorem 2.} 
		The cut rule is admissible in the extension of $\LK/\LJ$ with 
                synthetic rules corresponding to bipolar axioms.
	\end{block}


\end{frame}

%\section{The roadmap}
%
%\begin{frame}\frametitle{Obtaining rules from axioms}
%	
%	\scalebox{.8}{
%		\begin{tikzpicture}
%		\tikzstyle{operation}=[->,>=latex,thick]
%		\tikzstyle{state}=[rectangle,draw,thick, rounded corners=5pt, text width = 1.6cm, align=center]
%		\tikzstyle{etiquette}=[rectangle,draw,thick,dotted]
%		%%rectangles
%		%\draw [fill=red!20, thick] (5,-0.5) rectangle (8,-3.5);
%		%\draw [fill=blue!20, thick, rounded corners=7pt] (-1.5,-0.5) rectangle (1.5,0.5);
%		%%points
%		\uncover<1->{
%			\node[state, fill=gray!20] (i) at (0,0)  {Unpolarized\\Axiom};
%		}
%		\only<1,12>{
%			\node[right = of i] {$\forall x (((P_1(x) \impl P_2(x)) \wedge Q(x)) \impl\exists y R(x,y))$};
%		}
%		%
%		\uncover<2->{
%			\node[state, fill=blue!20] (ii) at (2.5,2) {Polarized\\Axiom};
%			\node[state, fill=blue!20] (ii') at (2.5,-2) {Polarized\\Axiom};
%			%	\draw[dotted,thick] (2.5,1.2)--(2.5,-1.2);
%		}
%		%
%		\uncover<2->{
%			\draw[operation] (i)--(ii) node[etiquette] at (0.7,1.2) {Polarizing};
%			\draw[operation] (i)--(ii');
%		}
%		\only<2-5,9,11>{
%			\draw[operation,dotted] (i)--(2,0.8);
%			\draw[operation,dotted] (i)--(2,0);
%			\draw[operation,dotted] (i)--(2,-0.8);
%		}
%		%
%		\only<2-4>{
%			\node[right] at (2,0.8) {$\negf{\bm\forall}x (((\posf{P_1}(x) \negf{\bm\impl} \negf{P_2}(x)) \posf{\bm\wedgep} \posf{Q}(x)) \negf{\bm\impl}\posf{\bm\exists}y \posf{R}(x,y))$};
%		}
%		\only<2-3,9>{
%			\node[right] at (2,0) {$\negf{\bm\forall}x (((\posf{P_1}(x) \negf{\bm\impl} \negf{P_2}(x)) \negf{\bm\wedgen} \negf{Q}(x)) \negf{\bm\impl}\posf{\bm\exists}y \posf{R}(x,y))$};
%		}
%		\only<2-3,11>{
%			\node[right] at (2,-0.8) {$\negf{\bm\forall}x (((\posf{P_1}(x) \negf{\bm\impl} \negf{P_2}(x)) \negf{\bm\wedgen} \posf{Q}(x)) \negf{\bm\impl}\posf{\bm\exists}y \posf{R}(x,y))$};
%		}
%		%
%		\uncover<3->{
%			\node[etiquette] at (4.5,3) {Is it bipolar?};
%		}
%		%
%		\uncover<4-8,9-10,12>{
%			\draw[thick] (ii)--(4.5,2) node[circle,draw,fill=green!40] {$\bm\checkmark$};
%		}
%		\uncover<11-12>{
%			\draw[thick] (ii')--(4.5,-2) node[circle,draw,fill=red!40] {$\bm\times$};%node[etiquette] 
%		}
%		%
%		\uncover<5-8,10,12>{
%			\node[state, fill=green!20] (iii) at (6.5,2) {Bipole\\in \LKF}; 
%			\draw[operation] (4.85,2)--(iii);
%		}
%		\only<5-7>{
%			\node[] at (8.5,-0.5) {$
%				\infer[\forall_l]{\jLf{\Gamma}{\negf{\bm\forall}x (((\posf{P_1}(x) \negf{\bm\impl} \negf{P_2}(x)) \posf{\bm\wedgep} \posf{Q}(x)) \negf{\bm\impl}\posf{\bm\exists}y \posf{R}(x,y))}{}{\Delta}}{
%					\infer[\impl_l]{{\jLf{\Gamma}{((P_1(t) \impl P_2(t)) \wedgep Q(t)) \impl \exists y R(t,y)}{}{\Delta}}}{
%						\infer[\wedgep_r]{\jRf{\Gamma}{(P_1(t) \impl P_2(t)) \wedgep Q(t)}{\Delta}}{
%							\infer[\krelease_r]{{\jRf{\Gamma}{P_1(t)\impl P_2(t)}{\Delta}}}{
%								\infer[\impl_r]{\jUnf{\Gamma}{\cdot}{P_1(t)\impl P_2(t)}{\Delta}}{
%									\infer[\kstore_l]{\jUnf{\Gamma}{P_1(t)}{P_2(t)}{\Delta}}{
%										\infer[\kstore_r]{\jUnf{\Gamma,P_1(t)}{\cdot}{P_2(t)}{\Delta}}{
%											\deduce{\jUnf{\Gamma,\posf{P_1(t)}}{\cdot}{\cdot}{\negf{P_2(t)},\Delta}}{}
%										}
%									}
%								}
%							} 
%							& 
%							\infer[\kinit_r]{\jRf{\Gamma}{\posf{Q(t)}}{\Delta}}{}
%						} 
%						& 
%						\infer[\krelease_l]{\jLf{\Gamma}{\exists y R(t,y)}{\Delta}}{
%							\infer[\exists_l]{\jUnf{\Gamma}{\exists y R(t,y)}{\cdot}{\Delta}}{
%								\infer[\kstore_l]{\jUnf{\Gamma}{R(t,z)}{\cdot}{\Delta}}{
%									\deduce{\jUnf{\Gamma, \posf{R(t,z)}}{\cdot}{\cdot}{\Delta}}{}
%								}
%							}
%						}
%					}
%				}
%				$};
%		}
%		%
%		\uncover<7-12>{
%			\node[state, fill=orange!20] (iv) at (9.5,2) {Inference\\rule for \LK};
%		}
%		\uncover<6-8,12>{
%			\draw[operation] (iii)--(iv) node[etiquette] at (8,3) {Synthesizing};
%		}
%		\only<7-8>{
%			\node[draw] at (11,1) {$
%				\infer[]{\Gamma = \Gamma', \emphdr{Q(t)}\seq\Delta}{
%					\Gamma, \emphdr{P_1(t)}\seq\emphdr{P_2(t)},\Delta
%					& 
%					\Gamma, \emphdr{R(t,z)}\seq\Delta}
%				$};
%		}
%		%
%		\only<10>{
%			\node[draw] at (10.5,1) {$
%				\infer[]{\Gamma\seq\Delta}{
%					\Gamma, \emphdr{P_1(t)}\seq\emphdr{P_2(t)},\Delta
%					&
%					\Gamma \seq\emphdr{Q(t)},\Delta
%					& 
%					\Gamma, \emphdr{R(t,z)}\seq\Delta}
%				$};
%		}
%		%
%		\only<12>{
%			\node[draw] at (11.5,0) {$
%				\infer[]{\Gamma = \Gamma', \emphdr{Q(t)}\seq\Delta}{
%					\Gamma, \emphdr{P_1(t)}\seq\emphdr{P_2(t)},\Delta
%					& 
%					\Gamma, \emphdr{R(t,z)}\seq\Delta}
%				$};
%			\node[draw] at (10.5,-1.5) {$
%				\infer[]{\Gamma\seq\Delta}{
%					\Gamma, \emphdr{P_1(t)}\seq\emphdr{P_2(t)},\Delta
%					&
%					\Gamma \seq\emphdr{Q(t)},\Delta
%					& 
%					\Gamma, \emphdr{R(t,z)}\seq\Delta}
%				$};
%		}
%		\end{tikzpicture}
%	}
%	
%	
%\end{frame}





\section{Examples}
%\subsection{Geometric axioms}
%
%\begin{frame}\frametitle{Geometric axioms as bipoles}
%	
%\only<1>{\emphdr{Geometric implication:}}
%%
%\only<2->{\emphdo{Polarized geometric implication:}}
%		
%%\begin{overlayarea}{\textwidth}{1cm}
%\only<1>{
%		\[\forall \overline{z} (P_1 \wedge \ldots \wedge P_m \, \impl \, \exists \overline{x}_1 M_1 \vee \ldots \vee \exists \overline{x}_n M_n ) 
%		\]
%}	
%\only<2>{
%\[\negf{\forall \overline{z}} (P_1^{\pm} \wedgepn \ldots \wedgepn P_m^{\pm} \, \negf{\impl} \, \posf{\exists \overline{x}_1} \hat{M_1} \veepn \ldots \veepn \posf{\exists \overline{x}_n} \hat{M_n} )
%		\]
%}
%\only<3>{
%	$$
%	\negf{\forall \overline{z}} (\posf{P_1^{+}} \posf{\wedgep} \ldots \posf{\wedgep} \posf{P_m^{+}} \, \negf{\impl} \, \posf{\exists \overline{x}_1} \hat{M_1} \veepn \ldots \veepn \posf{\exists \overline{x}_n} \hat{M_n} ) \, ,
%	$$
%}
%\only<4>{
%	$$
%	\negf{\forall \overline{z}} (\negf{P_1^{-}} \wedgepn \ldots \wedgepn \negf{P_m^{-}} \, \negf{\impl} \, \posf{\exists \overline{x}_1} \hat{M_1} \veepn \ldots \veepn \posf{\exists \overline{x}_n} \hat{M_n} ) \, ,
%	$$
%}
%%\end{overlayarea}
%
%\bigskip
%
%\begin{overlayarea}{\textwidth}{4cm}
%\only<1>{
%\begin{itemize}
%	\item $P_i$ atomic;
%	\item $M_j=Q_{j_1}\wedge\ldots\wedge Q_{j_{k_{j}}}$, $Q_{j_k}$ atomic; 
%	\item none of the variables in the vectors $\overline{x}_j$ are free in $P_i$.
%\end{itemize}	
%}
%\only<2>{
%\begin{itemize}
%\item $\posf{P_i^+}, \negf{P_i^-}$ atomic;
%\item $\hat{M_j} = Q_{j_1}^{\pm} \posf{\wedgep} \ldots \posf{\wedgep} Q_{j_{k_{j}}}^{\pm}$ , $Q_{j_k}^{\pm}$ atomic;
%\item none of the variables in the vectors $\overline{x}_j$ are free in $P_i$.
%\end{itemize}
%}		
%\only<3>{
%		\emph{Corresponding bipole}:	
%		$$\infer{\jUnf{\emphdr{\overline{P}},\Gamma'}{\null}{\null}{\Delta}}
%		{\jUnf{\emphdr{\overline{Q}_1[\overline{y}_1/\overline{x}_1]},\Gamma}{\null} {\null}{\Delta} & \ldots & \jUnf{\emphdr{\overline{Q}_n[\overline{y}_n/\overline{x}_n]},\Gamma}{\null} {\null}  {\Delta}}
%		$$
%		with $\overline{P}=\{\posf{P_i^+}\}, \overline{Q_j}=\{Q_{j_k}^\pm\}$. 
%
%		\bigskip
%				
%		\emphdb{Corresponding $\LK$ rule}:	
%		$$\infer[GRS]{\emphdr{\overline{P}},\Gamma'\vdash \Delta}
%		{\emphdr{\overline{Q}_1[\overline{y}_1/\overline{x}_1]},\Gamma\vdash\Delta & \ldots & \emphdr{\overline{Q}_n[\overline{y}_n/\overline{x}_n]},\Gamma\vdash\Delta}
%		$$
%	}
%	
%	\only<4>{
%		\emph{Corresponding bipole}:	
%		$$\infer[m+n\mbox{ premises}]{\jUnf{\Gamma}{\null}{\null}{\Delta}}
%		{\jUnf{\emphdr{\overline{Q}_j[\overline{y}_j/\overline{x}_j]},\Gamma}{\null} {\null}{\Delta} & \ldots &\jUnf{\Gamma}{\null} {\null}{\emphdr{P_i},\Delta} }
%		$$
%with $\overline{Q_j}=\{Q_{j_k}\}$. 
%
%		\bigskip
%				
%		\emphdb{Corresponding $\LK$ rule}:	
%		$$\infer[m+n\mbox{ premises}]{\Gamma\vdash \Delta}
%		{\emphdr{\overline{Q}_j[\overline{y}_j/\overline{x}_j]},\Gamma\vdash \Delta & \ldots & \Gamma\vdash \emphdr{P_i},\Delta}
%		$$		
%		}
%\end{overlayarea}
%	
%\end{frame}


\begin{frame}\frametitle{\only<1-4>{Example: Horn clauses as bipoles} \only<5>{Other examples}}

\only<1>{\[
		\forall \overline{z}(P_1 \wedge \ldots \wedge P_m \, \impl \, Q)
		\]}
	
\only<2>{\[
		\negf{\forall \overline{z}}(P_1^{\pm} \wedgepn \ldots \wedgepn P_m^{\pm} \, \negf{\impl} \, Q^{\pm})
		\]}
	
\only<3>{\[
	\negf{\forall \overline{z}}(\posf{P_1^{+}} \posf{\wedgep} \ldots \posf{\wedgep} \posf{P_m^{+}} \, \negf{\impl} \, \posf{Q^{+}} )
	\]}

\only<4>{\[
	\negf{\forall \overline{z}}(\negf{P_1^{-} \wedgen} \ldots \negf{\wedgen P_m^{-}} \, \negf{\impl} \, \negf{Q^{-}})
	\]}
		
\bigskip


\begin{overlayarea}{\textwidth}{2cm}
\only<3>{
$$	\infer[FC]
	{\emphdr{\overline{P}}, \Gamma'\vdash\Delta}
	{\emphdr{Q}, \Gamma\vdash\Delta}	
	$$
\begin{center}
\emph{Forward-chaining} \\
\emph{[Simpson, Negri, Ciabattoni]}
\end{center}
}

\only<4>{
$$\infer[BC]
	{\Gamma\vdash \emphdr{Q},\Delta'}
	{\Gamma\vdash \emphdr{P_1}, \Delta & \ldots &\Gamma\vdash \emphdr{P_m}, \Delta}
	$$
\begin{center}
\emph{Back-chaining}\\
\emph{[Vigan\`o]}
\end{center}
}
\only<5>{
\begin{center}
Geometric, co-geometric, universal axioms...
\end{center}
\medskip
\[\negf{\forall \overline{z}} (P_1^{\pm} \wedgepn \ldots \wedgepn P_m^{\pm} \, \negf{\impl} \, \posf{\exists \overline{x}_1} \hat{M_1} \veepn \ldots \veepn \posf{\exists \overline{x}_n} \hat{M_n} )
		\]
		\smallskip
\[
		\negf{\forall \overline{z}} (\negf{\forall \overline{x}_1} \hat{M_1} \wedgepn \ldots \wedgepn \negf{\forall \overline{x}_n} \hat{M_n}  \, \negf{\impl} \, \negf{P_1^{-} \veen} \ldots \negf{\veen P_m^{-}} ) 
		\]
		\smallskip
\[
		\negf{\forall \overline{z}}(P_1^{\pm} \wedgepn \ldots \wedgepn P_m^{\pm} \, \negf{\impl} \, Q_1^{\pm} \veepn \ldots \veepn Q_n^{\pm})
		\]
}
\end{overlayarea}
\end{frame}


\begin{frame}
\frametitle{Recapitulation}

\begin{itemize}

\item \textbf{Polarity of connectives:} invertibility vs
  non-invertibility of introduction rules\\[10pt]

\item \textbf{Focusing:} uses polarity to organize proofs into a
  two-phase structure.\\[10pt]

  These features of proofs arose within linear logic.  The \LKF and
  \LJF proof system apply these features to classical and
  intuitionistic logics.\\[10pt]

\item \textbf{Synthetic inference rules:}
\pause
  \begin{itemize}
  \item \textbf{Bipoles:} A flexible means exists to translate bipoles
    ($\mathcal{N}_2$) to inference rules involving only atomic
    formulas: see Marin, Miller, Pimentel, \& Volpe [2022].\\[6pt]
\pause

  \item \textbf{Non-bipoles:} The topic of the rest of this talk.
    \begin{itemize}
    \item Transform non-bipoles into bipoles.
    \item Use a higher-level system of rules.
    \end{itemize}
  \end{itemize}

\end{itemize}

\end{frame}


\begin{frame}
\frametitle{One approach to treating non-bipoles: Remove them}

Get rid of non-bipolar formulas by introducing new predicate symbols.

\begin{itemize}
\item Tseitin [1960's], Mints et al.~[1982].
\item Andreoli: skolemization [1992], bipolarization [2001].
\item Dyckhoff \& Negri: geometrisation [2015]
\item See Dyckhoff \& Negri for many other names and references.
\end{itemize}

\vfill\pause
With higher-order quantification, provability can be maintained.
\[
  (u\impl (p\impl q)\impl r) \impl s \dashv\vdash \exists x. 
\left(\begin{array}{c}
    (u\impl(x\impl r) \impl s) \wedge~\\
    (x\impl (p\impl q))  \wedge~\\
    ((p\impl q)\impl x)
\end{array}
\right)
\]

If you drop $\exists x$ for a new predicate symbol, the expressions
are equisatisfiable.

\newcommand{\sledom}{\Relbar\joinrel\mathrel{|}}

\[
  (u\impl (p\impl q)\impl r) \impl s \sledom \models
\left(\begin{array}{c}
    (u\impl (x\impl r) \impl s) \wedge~\\
    (x\impl (p\impl q))  \wedge~\\
    ((p\impl q)\impl x)
\end{array}
\right)
\]
\vfill
With only implications, $B$ is of order $n$ if and only if $B\in\mathcal{N}_n$.
\vfill
\end{frame}

\begin{frame}
\frametitle{Another approach to treating non-bipoles: Higher-level of rules}

Let $C$ denote $(u\impl (p\impl q)\impl r) \impl s)$.  (Assume that
$s$ has negative polarity.)

\[
  \infer{\jUnf{\Gamma, C}{\cdot}{\cdot}{s}}{
  \infer={\jLf{\Gamma, C}{(u\impl (p\impl q)\impl r) \impl s}{s}}{
    \infer[\krelease_r]{\jRf{\Gamma,C}{u}}{
       \infer[\kstore_l]{\jUnf{\Gamma,C}{\cdot}{u}{\cdot}}
                        {\jUnf{\Gamma,C}{\cdot}{\cdot}{u}}}
& 
  \infer[\krelease_l]{\jRf{\Gamma,C}{(p\impl q)\impl r}}{
  \infer{\jUnf{\Gamma,C}{\cdot}{(p\impl q)\impl r}{\cdot}}{
  \infer{\jUnf{\Gamma,C}{p\impl q}{r}{\cdot}}{
\jUnf{\Gamma,C,p\impl q}{\cdot}{\cdot}{r}
}}}
& 
    \infer[\kinit_l]{\jLf{\Gamma,C}{s}{s}}{}
}}
\]
This justifies the synthetic inference rule
\[
  \infer[C]{\Gamma\seq s}{ \Gamma\seq u & \Gamma,p\impl q \seq r}
\]
Unfortunately, this contains an occurrence of a logical connective.
\vfill
Of course, the $\mathcal{N}_1$ formula $p\impl q$ formula can be
replaced by an inference rule.  This rule depends on polarity, of
course. 
\vfill
\end{frame}

\begin{frame}
\frametitle{Higher-level of rules: an example}

The synthetic rule for $C=(u\impl (p\impl q)\impl r) \impl s)$ (where 
$s$ has negative polarity).
\[\left[\quad\vcenter{
  \infer[C]{\Gamma\seq s}{\Gamma\seq u
   \quad &\quad
   \infer{\Gamma\seq r}{
     \deduce{\strut\vdots}{
     \left(\vcenter{\hbox{Rule based on $p\impl q$}}\right)}}}
}\quad\right]\]
The second premise has an inference rule that is
available to prove that premise.\pause  The shape of that rule depends on 
the polarity of $p$ and $q$.  There are four possibilities.
\[\qquad
  \infer[(p-,~q-)]{\Psi\seq q}{\Psi\seq p}
  \qquad 
  \infer[(p-,~q+)]{\Psi\seq E}{\Psi\seq p & \Psi,q \seq E}
\]
\[
  \infer[(p+,~q-)]{\Psi,p\seq q}{}
  \qquad 
  \infer[(p+,~q+)]{\Psi,p\seq E}{\Psi,p, q \seq E}
\]
For example,
\[\left[\quad\vcenter{
  \infer[C]{\Gamma\seq s}{\Gamma\seq u
   \quad &\quad
   \infer{\Gamma\seq r}{
     \deduce{\strut\vdots}{
     \left(\vcenter{\infer{\Psi,p\seq E}{\Psi,p, q \seq E}}\right)}}}
  }\quad\right]
\]
\end{frame}

\newcommand{\Twoseq}[3]{{\color{darkred}#1:}\,#2\vdash #3}
\newcommand{\lub   }[3]{\textsf{lub}~#1~#2~#3}

\begin{frame}
\frametitle{Higher-level of rules: An example with quantifiers}

The (polarized) formula stating the existence of least upper bounds.
\[
  \forall x\forall y\exists z(x\le z\wedgep y\le z\wedgep
              \forall w(x\le w\wedgep y\le w\impl z\le w)),
\]
Focusing on this formula yields the derivation.
\[
  \infer
        {\Twoseq{\Sigma}{\Gamma}{\Delta}}
        {\Twoseq{\Sigma,z}
                {\forall w(x\le w\wedgep y\le w\impl z\le w),~
                   x\le z,~y\le z,~\Gamma}
                {\Delta}}
\]
Sequents are prefixed with a list of eigenvariables $\Sigma$ which are
bound over the sequent.
\vfill

The assumption $\forall w(x\le w\wedgep y\le w\impl z\le w)$ can be
converted to an inference rule (depending on the polarity of the $\le$
predicate).  For example,
\[
  \infer{\Twoseq{\Sigma}{\Gamma}{\Delta}}
        {\infer{\Twoseq{\Sigma,z}{\Gamma,x\le z,~y\le z}{\Delta}}
               { \deduce{\strut\vdots}{
     \left(\vcenter{
  \infer{\Twoseq{\Sigma,z}{\Gamma}{z\le w}}
        {\Twoseq{\Sigma,z}{\Gamma}{x\le w} \qquad
         \Twoseq{\Sigma,z}{\Gamma}{y\le w}}
     }\right)}}}
\]
There are no logical constant.  The \emph{scope} of variables is
getting complicated.
\end{frame}

\begin{frame}
\frametitle{Higher-level of rules: Continued}

The $\mathcal{N}_3$ formula\quad
$
  \forall x\forall y\exists z(x\le z\wedgep y\le z\wedgep
              \forall w(x\le w\wedgep y\le w\impl z\le w)),
$
can be bipolarized by introducing a new predicate ($\lub{x}{y}{z}$
denotes $z$ is the least upper bound of $x$ and $y$).
\[
  \forall x\forall y\exists z.
  \left[
  \begin{array}{c}
    (x\le z\wedgep y\le z\wedgep\lub{x}{y}{z})\wedgen~\\
    \lub{x}{y}{z}\equiv \forall w.(x\le w\wedgep y\le w\impl z\le w))
  \end{array}
  \right]
\]

Moving quantifiers  **** CONTINUE HERE ***
\[
  \begin{array}{c}
    \forall x\forall y\exists z.[(x\le z\wedgep y\le z\wedgep\lub{x}{y}{z})]
      \wedgen~\\
     \lub{x}{y}{z}\equiv \forall w.(x\le w\wedgep y\le w\impl z\le w))
  \end{array}
  \]

%% Focusing on this formula yields the derivation.
%% \[
%%   \infer
%%         {\Twoseq{\Sigma}{\Gamma}{\Delta}}
%%         {\Twoseq{\Sigma,z}
%%                 {\forall w(x\le w\wedgep y\le w\impl z\le w),~
%%                    x\le z,~y\le z,~\Gamma}
%%                 {\Delta}}
%% \]
%% Sequents are prefixed with a list of eigenvariables $\Sigma$ which are
%% bound over the sequent.
%% \vfill

%% The assumption $\forall w(x\le w\wedgep y\le w\impl z\le w)$ can be
%% converted to an inference rule (depending on the polarity of the $\le$
%% predicate).  For example,
%% \[
%%   \infer{\Twoseq{\Sigma}{\Gamma}{\Delta}}
%%         {\infer{\Twoseq{\Sigma,z}{\Gamma,x\le z,~y\le z}{\Delta}}
%%                { \deduce{\strut\vdots}{
%%      \left(\vcenter{
%%   \infer{\Twoseq{\Sigma,z}{\Gamma}{z\le w}}
%%         {\Twoseq{\Sigma,z}{\Gamma}{x\le w} \qquad
%%          \Twoseq{\Sigma,z}{\Gamma}{y\le w}}
%%      }\right)}}}
%% \]
%% There are no logical constant.  The \emph{scope} of variables is
%% getting complicated.
\end{frame}

\begin{frame}
\frametitle{Higher-level of rules in natural deduction}

If we limit ourselves to negative connectives and negative atoms, then
the sequent calculus is essentially natural deduction.

Peter Schroeder-Heister, Arvon

Edinburgh LF
\end{frame}
\end{document}
