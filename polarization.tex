%!TEX root = main.tex
% !TEX spellcheck = en-US

\LKF and \LJF ~\cite{LiaMil09}can accommodate both the $Q$ and $T$ protocols as well
as a mix of those protocols.
%
The proof system \LKF, for first-order classical logic, and the
proof system \LJF, for first-order intuitionistic logic, are
presented in Figures~\ref{fig:lkf} and~\ref{fig:ljf}, respectively.

In order to obtain their flexibility in capturing various focusing
regimes, the \LKF and \LJF proof systems use \emph{polarized}
formulas.
%
A \emph{polarized classical (first-order) formula} is a formula built
using atomic formulas, 
the usual first-order quantifiers $\forall$ and $\exists$, the implication
$\impl$, and polarized versions of the logical connectives and
constants, i.e., $\ntrue$, $\ptrue$,$\nfalse$, $\pfalse$, $\veen$,
$\veep$, $\wedgen$, $\wedgep$.
%
A \emph{polarized intuitionistic (first-order) formula} is a polarized
classical formula in which the logical connectives $\nfalse$ and
$\veen$ do not occur.
%
The positive and negative versions of connectives and constants
have identical truth conditions but different inference rules
inside the polarized proof systems.
%
For example, the left introduction rule for $\wedgep$ is invertible
while the left introduction rule for $\wedgen$ is not invertible.

%We shall also find it necessary to use \emph{delays}: if $B$ is a
%polarized formula then we define $\negd{B}$ to be (the always
%negative) $B\wedgen\ntrue$ and $\posd{B}$ to be (the always positive)
%$B\wedgep\ptrue$.
%%
%Equivalently, we can take $\posd{\cdot}$ to be the 1-ary version of
%either the binary $\veep$ or $\wedgep$ and
%take $\negd{\cdot}$ to be the 1-ary version of either
%the binary $\veen$ or $\wedgen$.
%%
%(The 0-ary version of these four connectives correspond to the
%logical units $\pfalse$, $\ptrue$, $\nfalse$, 
%%or 
%$\ntrue$.)

If a formula's top-level connective is $\ptrue$,
$\pfalse$, $\veep$, $\wedgep$, or $\exists$, then that formula is
\emph{positive}.
%
If a formula's top-level connective is $\ntrue$, $\nfalse$,
$\veen$, $\wedgen$, $\impl$, or $\forall$, then it is \emph{negative}.
%
Note that in the intuitionistic system \LJF, we have only one
disjunction and one falsum, both of which exist only with positive
polarity.
%
The way to form the negation of the polarized formula $B$ is with the
formula $B\impl\pfalse$: this formula has negative polarity no matter
the polarity of $B$.

%\ednote{MV: We said above that $\nfalse$ does not occur in the
%intuitionistic language. EP. I think that it should be $\pfalse$.} 

In both \LKF and \LJF, every polarized formula is classified as
positive or negative.
%
This means that we must also provide a polarity to atomic formulas.
%
As it turns out, this assignment of polarity to atomic formulas can,
in principle, be arbitrary.
%
%In particular, an \emph{atomic bias assignment} is a function
%$\bias{\cdot}$ that maps atomic formulas to the set of two tokens
%$\{pos,~neg\}$: if $\bias{A}$ is $pos$ then that atomic formula is
%positive and if $\bias{A}$ is $neg$ then that atomic formula is
%negative.
%%
%We may ask that all atomic formulas are positive, that
%they are all negative, or we can mix polarity
%assignments. 
%%
%In particular, the atomic bias assignment $\biasp{\cdot}$ assigns all
%atoms a positive polarity while $\biasn{\cdot}$ assigns all
%atoms a negative polarity.
%%
%We shall assume that an atomic bias
%assignment is also \emph{stable under substitution}: that is, for all
%substitutions $\theta$,  $\bias{\theta A} = \bias{A}$.
%%
%In first-order logic, this is equivalent to saying that such bias
%assignments are \emph{predicate determined}: that is, if atoms $A$ and
%$A'$ have the same predicate head, then $\bias{A}=\bias{A'}$.

%We say that the pair $\tup{\delta,\hat B}$ is a \emph{polarization of}
%$B$ if $\bias{\cdot}$ is an atomic bias assignment and if every
%occurrence of $\true$, $\wedge$, $\false$, and $\vee$ in $B$ is
%labeled with either the $+$ or $-$ annotation.
%%
%If $B$ has $n$ occurrences of these logical connectives then there are
%$2^n$ different ways to place these $+$ or $-$ symbols.
%%
%We shall also allow the insertion of any number of $\posd{\cdot}$ and
%$\negd{\cdot}$ into $\hat B$ as well.
%%
%In other words, the polarized formula $\tup{\delta,C}$ is a
%polarization of $B$ if deleting all delays and all $+$ and $-$
%annotations on logical connectives of $C$ results in $B$.
%
Note that we use $\impl$, $\forall$, and $\exists$ in both unpolarized
as well as polarized formulas: we can do this since the polarity of
these connectives is not ambiguous.
%
In classical logic, the polarity of $\true$, $\wedge$, $\false$, and
$\vee$ is ambiguous and all of these can be positive or negative.
%
In intuitionistic logic, only the polarity of $\true$ and $\wedge$
is ambiguous.
%
In both of these logics, however, the polarity of atoms is equally
ambiguous.
%
%Finally, if $\tup{\delta,\hat B}$ is a polarization of $B$, we shall
%generally drop explicit reference to $\delta$ and simply say that
%$\hat B$ is a polarization of $B$: often, the atomic bias assignment
%is either not important or can be inferred from context.


\subsection{Focused proof systems}
\label{sec:focused ps}


\begin{figure}
	{\sc Asynchronous Rules}
\[
  \infer[\kern-2pt\veen_r]{\jUnf{\Gamma}{\Sigma}{A \veen\kern-3pt B, \Omega}{\Delta}}
                 {\jUnf{\Gamma}{\Sigma}{A,B, \Omega}{\Delta}}   
  \qquad 
  \infer[\kern-2pt\wedgen_r]{\jUnf{\Gamma}{\Sigma}{A \wedgen B, \Omega}{\Delta}}
                   {\jUnf{\Gamma}{\Sigma}{A, \Omega}{\Delta}
                    & 
                    \jUnf{\Gamma}{\Sigma}{B, \Omega}{\Delta}}
\]
\[
  \infer[\wedgep_l]{\jUnf{\Gamma}{A\wedgep B,\Sigma}{\Omega}{\Delta}}
                   {\jUnf{\Gamma}{A , B, \Sigma}{\Omega}{\Delta}}
  \qquad
  \infer[\veep_l]{\jUnf{\Gamma}{A\veep B,\Sigma}{\Omega}{\Delta}}
                 {\jUnf{\Gamma}{A,\Sigma}{\Omega}{\Delta}
                  & 
                  \jUnf{\Gamma}{B,\Sigma}{\Omega}{\Delta}}
\]
\[
  \infer[\forall_r]{\jUnf{\Gamma}{}{\forall x.B, \Omega}{\Delta}}
                   {\jUnf{\Gamma}{}{[y/x]B, \Omega}{\Delta}}	
  \qquad
  \infer[\exists_l]{\jUnf{\Gamma}{\exists x.B, \Sigma}{\Omega}{\Delta}}
                   {\jUnf{\Gamma}{[y/x]B,\Sigma}{\Omega}{\Delta}}
\]
\[
  \infer[\kern-2pt\impl_r]{\jUnf{\Gamma}{\Sigma}{A \impl B, \Omega}{\Delta}}
                 {\jUnf{\Gamma}{\Sigma,A}{B,\Omega}{\Delta}}
  \qquad
  \infer[\kern-2pt\ptrue_l]{\jUnf{\Gamma}{\ptrue\kern-2pt, \Sigma}{\Omega}{\Delta}}
                  {\jUnf{\Gamma}{\Sigma}{\Omega}{\Delta}}
  \qquad
  \infer[\kern-2pt\nfalse_r]{\jUnf{\Gamma}{\Sigma}{\nfalse\kern-2pt,\Omega}{\Delta}}
                            {\jUnf{\Gamma}{\Sigma}{\Omega}{\Delta}}
\]
\[
  \infer[\ntrue_r]{\jUnf{\Gamma}{\Sigma}{\ntrue\kern-3pt,\Omega}{\Delta}}{}
  \qquad
  \infer[\kern-2pt\pfalse_l]{\jUnf{\Gamma}{\pfalse\kern-2pt,\Sigma}{\Omega}{\Delta}}{}
\]
	
\medskip{\sc Synchronous Rules}	

\[ 
  \infer[\impl_l]{\jLf{\Gamma}{A\impl B}{\Delta}}
                 {\jRf{\Gamma}{A}{\Delta} &  \jLf{\Gamma}{B}{\Delta}}
  \quad	
  \infer[\veen_l]{\jLf{\Gamma}{A \veen\kern-3pt B}{\Delta}}
                 {\jLf{\Gamma}{A}{\Delta}
                  & 
                  \jLf{\Gamma}{B}{\Delta}}
  \quad
  \infer[\kern -2pt\wedgen_l]{\jLf{\Gamma}{A_1 \wedgen\kern-3pt A_2}{\Delta}}
                   {\jLf{\Gamma}{A_i}{\Delta}}
\]
\[
  \infer[\wedgep_r]{\jRf{\Gamma}{A \wedgep B}{\Delta}}
                   {\jRf{\Gamma}{A}{\Delta}
                    & 
                    \jRf{\Gamma}{B}{\Delta}}
  \qquad
  \infer[\veep_r]{\jRf{\Gamma}{A_1 \veep A_2}{\Delta}}
                 {\jRf{\Gamma}{A_i}{\Delta}}
\]
\[
  \infer[\forall_l]{\jLf{\Gamma}{\forall x.B}{\Delta}}
                   {\jLf{\Gamma}{[t/x]B}{\Delta}}
  \quad
  \infer[\exists_r]{\jRf{\Gamma}{\exists x.B}{\Delta}}
                   {\jRf{\Gamma}{[t/x]B}{\Delta}}
  \quad
  \infer[\ptrue_r]{\jRf{\Gamma}{\ptrue}{\Delta}}{}
  \quad
  \infer[\nfalse_l]{\jLf{\Gamma}{\nfalse}{\Delta}}{}
\]

\medskip{\sc Identity rules}
	
\[
  \infer[\kinit_l]{\jLf{\Gamma}{N_a}{N_a, \Delta}}{}
  \qquad
  \infer[\kinit_r]{\jRf{\Gamma,P_a}{P_a}{\Delta}}{}
%	\qquad 
%	\infer[cut]{\jUnf{\Gamma}{}{}{\Delta}}
%	{\jUnf{\Gamma}{}{F}{\Delta}
%	& 
%	\jUnf{\Gamma}{F}{}{\Delta}}
\]
	
\medskip{\sc Structural rules}
	
\[
  \infer[\kdecide_l]{\jUnf{\Gamma,N}{}{}{\Delta}}
                    {\jLf{\Gamma,N}{N}{\Delta}}
  \quad
  \infer[\kdecide_r]{\jUnf{\Gamma}{}{}{P, \Delta}}
                    {\jRf{\Gamma}{P}{P, \Delta}}
  \quad
  \infer[\krelease_l]{\jLf{\Gamma}{P}{\Delta}}
                     {\jUnf{\Gamma}{P}{}{\Delta}}
  \quad
  \infer[\krelease_r]{\jRf{\Gamma}{N}{\Delta}}
                     {\jUnf{\Gamma}{}{N}{\Delta}}
\]
\[
  \infer[\kstore_l]{\jUnf{\Gamma}{C,\Sigma}{\Omega}{\Delta}}
                   {\jUnf{C,\Gamma}{\Sigma}{\Omega}{\Delta}}
  \qquad
  \infer[\kstore_r]{\jUnf{\Gamma}{}{D, \Omega}{\Delta}}
                   {\jUnf{\Gamma}{}{\Omega}{D, \Delta}}
\]
	
Here, $P$ is positive, $N$ is negative, $C$ is a negative formula or
positive atom, $D$ a positive formula or negative atom, $N_a$ is
a negative atom, and $P_a$ is a positive atom.  Other formulas
are arbitrary. In the rules $\forall_r$  and $\exists_l$ the
eigenvariable $y$ does not occur free in any formula of the
conclusion.
\caption{The focused classical sequent calculus \LKF.}
\label{fig:lkf}
\end{figure}	

\begin{figure}
{\sc Asynchronous Rules}

\[
  \infer[\impl_r]{\jUnfG{\Sigma}{A \supset B}}{\jUnfG{A,\Sigma}{B}}   
  \qquad
  \infer[\wedgen_r]{\jUnfG{\Sigma}{A \wedgen B}}
                   {\jUnfG{\Sigma}{A} \quad \jUnfG{\Sigma}{B}}
\]
\[
  \infer[\wedgep_l]{\jUnfGamb{A\wedgep B,\Sigma}}{\jUnfGamb{A , B, \Sigma}}
  \qquad
  \infer[\veep_l]{\jUnfGamb{A\veep B,\Sigma}}
        {\jUnfGamb{A,\Sigma}\quad \jUnfGamb{B,\Sigma}}
\]
\[
  \infer[\forall_r]{\jUnfG{\Sigma}{\forall x.B}}{\jUnfG{\Sigma}{[y/x]B}}
  \qquad
  \infer[\exists_l]{\jUnfGamb{\exists x.B, \Sigma}}{\jUnfGamb{[y/x]B,\Sigma}}
 \]
\[
  \infer[\ntrue_r]{\jUnfG{\Sigma}{\ntrue}}{}\qquad
  \infer[\ptrue_l]{\jUnfGamb{\ptrue, \Sigma}}{\jUnfGamb{\Sigma}}
   \qquad
  \infer[\pfalse_l]{\jUnfGamb{\pfalse, \Sigma}}{}
\]

\medskip{\sc Synchronous Rules}

\[ 
  \infer[\impl_l]{\jLfG{A \impl B}}{  \jRfG{A}\quad \jLfG{B}}
  \qquad
  \infer[\veep_r]{ \jRfG{A_1 \veep A_2}}{\jRfG{A_i}}
  \qquad
  \infer[\wedgen_l]{\jLfG {A_1 \wedgen A_2}}{\jLfG{A_i}}
\]
\[
  \infer[\wedgep_r]{\jRfG{A \wedgep B}}{\jRfG{A} \quad \jRfG{B}}
  \quad
  \infer[\forall_l]{\jLfG{\forall x.B}}{\jLfG{[t/x]B}}
  \quad
  \infer[\exists_r]{\jRfG{\exists x.B}}{\jRfG{[t/x]B}}
  \quad
  \infer[\ptrue_r]{\jRfG{\ptrue}}{}
\]

\medskip{\sc Identity rules}

\[
  \infer[\kinit_l]{\jLf{\Gamma}{N_a}{N_a}}{}
  \qquad
  \infer[\kinit_r]{\jRf{\Gamma,P_a}{P_a}}{}
  \qquad
%  \infer[cut]{\jUnf{\Gamma}{}{}{R}}
%             {\jUnf{\Gamma}{}{F}{}\qquad \jUnf{\Gamma}{F}{}{R}}
\]

\medskip{\sc Structural rules}

\[
  \infer[\kern -2pt \kdecide_l]{\jUnf{\Gamma,N}{}{}{R}}{\jLf{\Gamma,N}{N}{R}}
  \quad
  \infer[\kern -2 pt \kdecide_r]{\jUnf{\Gamma}{}{}{P}}{\jRf{\Gamma}{P}}
  \quad
  \infer[\krelease_l]{\jLf{\Gamma}{P}{R}}{\jUnf{\Gamma}{P}{}{R}}
  \quad
  \infer[\krelease_r]{\jRf{\Gamma}{N}}{ \jUnf{\Gamma}{}{N}{}}
\]
\[
\infer[\kstore_l]{\jUnfamb{\Gamma}{C,\Sigma}{}}
                 {\jUnfamb{C,\Gamma}{\Sigma}{}}
  \qquad
  \infer[\kstore_r]{\jUnf{\Gamma}{}{D}{}}{\jUnf{\Gamma}{}{}{D}}
\]

Here, $P$ is positive, $N$ is negative, $C$ is a negative formula or
positive atom, $D$ a positive formula or negative atom, $N_a$ is a
negative atom, and $P_a$ is a positive atom.  Other formulas are
arbitrary. $\Rscr$ denotes $\Delta_1 \Uparrow \Delta_2$ where the
union of $\Delta_1$ and $\Delta_2$ contains at most one formula.  
In the rules $\forall_r$  and $\exists_l$ the
eigenvariable $y$ does not occur free in any formula of the
conclusion.
\caption{The focused intuitionistic sequent calculus \LJF.}
\label{fig:ljf}
\end{figure}	

The inference rules of \LKF and \LJF presented in Figures~\ref{fig:lkf} and~\ref{fig:ljf}, respectively, involve three kinds of sequents 
\begin{itemize}
\item $\jUnf{\Gamma}{\Sigma}{\Omega}{\Delta}$ belongs to the {\em asynchronous phase}. During this phase, all negative (resp. positive) formulas in  $\Omega$ (resp. $\Sigma$) are introduced and all atoms and positive (resp. negative) formulas  are stored in the context $\Delta$ (resp. $\Gamma$);
\item $\jLf{\Gamma}{B}{\Delta},\quad\hbox{and}\quad
  \jRf{\Gamma}{B}{\Delta}$ belongs to the {\em synchronous phase}, where all positive connectives at the root of $B$ are introduced. 
\end{itemize}
Here $\Gamma$, $\Sigma$, $\Omega$ and $\Delta$ are
multisets of polarized formulas and $B$ is a polarized formula.
%
The formula occurrence $B$ in a $\Downarrow$-sequent is called the
\emph{focus} of that sequent.

The dynamics of the different phases is detailed below.

\noindent
{\bf Asynchronous Phase.} Asunchronous rules can be applied eagerly, in any order.
This process includes storing formulas: note that the rules $\kstore_r, \kstore_l$ move, to the 
contexts $\Gamma/\Delta$, atomic or negative/positive formulas since they cannot be decomposed during the asynchronous phase.

The asynchronous phase ends when $\Sigma,\Omega$ in unfocused sequents are empty. A decision rule $\mathsf{D_l}$ or  $\mathsf{D_r}$ is then applied,  and
a synchronous phase starts by {\em focusing} on a non-atomic formula, either negative on the left or positive on the right.

\noindent
{\bf Synchronous Phase.} Once we focus on a formula, the proof  follows by applying synchronous rules, where  the focus persists on the decomposed subformulas until either: the proof ends with an instance of an axiom;
a negative (resp. positive) formula  on the right (resp. left) is reached, and the synchronous phase  ends with the application of one of the release rules $\krelease_r,\krelease_l$.


There are two initial rules, $\kinit_r,\kinit_l$, acting on right positive atoms and left negative atoms, respectively, meaning that derivations should necessarily end when focusing on such formulas.


The system  \LJF is depicted in a separate figure for the sake of clarity.
%
However, one can notice that, similarly to what we have for \LJ and
\LK in the original Gentzen formulations, \LJF can be seen as a
restriction of \LKF, where the rules for $\nfalse$ and $\veen$ are
omitted and only one formula is admitted in the succedent of sequents.  In
particular, this implies that $(i)$ in the left rule for $\impl$, the
right context of the conclusion is not present in the left premise;
$(ii)$ in the rule $\kdecide_r$, the formula placed under focus is
not subjected to contraction; and $(iii)$ a sequent of the form $\jRf{\Gamma}{B}{\Delta}$, when used in an
\LJF proof, is such that $\Delta$ is empty.
%
In that case, we write that sequent as simply $\jRf{\Gamma}{B}{}$.